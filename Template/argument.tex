\section{Argument}

\subsection{Subjective Evidence}

\subsection{Time Extrapolation}
The author claims that \(\cdots\), because ({\color{red}Point out}).
This assumption is unwarranted because things rarely remain the same over the extended period of time ({\color{red}Reason}).
There are likely all kinds of difference between (past) and (present).
For example, ({\color{red}Analysis}).
And of these scenarios, if true, would serve to undermine the claim that \(\cdots\) ({\color{red}Evaluation}).

\subsection{Comparison}
\paragraph{Universal}
The arguer's recommendation relies on what might be a poor analogy between A and B ({\color{red}Point out}).
The analogy falsely depends on the assumption that \(\cdots\) in both A and B is similar.
({\color{red}Analysis}).
In short, without accounting for such possible differences between A and B, the arguer cannot prove that B will reap the similar benefits from the proposed method ({\color{red}Evaluation}).

\paragraph{Individual vs. Group}
One problem with the argument is that it assumes that the nationwide statistics about \(\cdots\) apply equally to \(\cdots\) ({\color{red}Point out}).
Yet this might not be the case, for a variety of possible reasons ({\color{red}Reason}).
Perhaps \(\cdots\); or perhaps \(\cdots\) ({\color{red}Analysis}).
Without ruling out such posibilities, the author cannot justifiably conclude that \(\cdots\) ({\color{red}Evaluation}).

\paragraph{Average}


\subsection{Causality}
\paragraph{Non-simultaneity}
The arguer fails to establish the causal relationship between A and B ({\color{red} Point out}).
It's highly possible that other factors contribute to B ({\color{red} Reason}).
For instance, B might have resulted from C.
It is also likely that D caused B ({\color{red} Analysis}).
Lacking evidence that links A to B, it is presumptuous to suggest that A was responsible for B ({\color{red} Evaluation}).

\paragraph{Simultaneity}
Based on the fact that A occurred after/with B the editor infers that B should be responsible for A ({\color{red} Point out}).
However, the sequence of these events, in itself, does not suffice to prove that the earlier development caused the later one/another ({\color{red} Reasom}).
It might have resulted from the other reasons instead: C, D, or E, to name a few ({\color{red} Analysis}), without ruling out scenarios such as these, the editor cannot establish a cause-and-effect relationship between A and B upon which the editor's recommendation depends ({\color{red} Evaluation}).

\subsection{Profitability}
The author's conclusion that \(\cdots\) is unwarranted ({\color{red} Point out}).
Profit is a factor relating to not only revenue, but also cost ({\color{red} Reason}).
It's entirely possible that the cost of A or other costs associated with B, C will offset, even outweigh the revenue.
Besides, a myriad of other unexpected occurences, such as unfavorable economic depression, might present \(\cdots\) from being as profitable as the argument predicts ({\color{red} Analysis}).

\paragraph{Lost and gain}
The author's conclusion that \(\cdots\) is unwarranted ({\color{red} Point out}), because he (or she) fails to weigh the advantage and disadvantage about \(\cdots\) ({\color{red} Reason}).
It's entirely possible that the advantages of A or other benefits associated with B, C will offset, even outweigh the disadvantages ({\color{red} Analysis}).