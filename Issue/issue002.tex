\chapter{The best way for a society to prepare its young people for leadership in government, industry, or other fields is by instilling in them a sense of cooperation, not competition}

\section{Score 6}
Whenever people argue that history is a worthless subject or that there is nothing to be gained by just ``memorizing a bunch of stupid names and dates,'' I simply hold my tongue and smile to myself.
What I'm thinking is that, as cliche as it sounds, you do learn a great deals from history (and woe to those who fail to learn those lessons).
It is remarkable to think of the number of circumstances and situations in which even the most \emph{rudimentary} knowledge of history will turn out to be invaluable.
Take, for example, the issue at hand here.
IS it better for a society to instill in future leaders a sense of competition or cooperation?
Those who have not examined leaders throughout time and across a number of fields might not have the ability to provide a thorough and convincing answer to this question, in spite of the fact that it is crucial to the future functioning of our society.
Looking closely at the question of leadership and how it has worked in the past, I would have to agree that the best way to prepare young people for leadership is to instill in them a sense of cooperation.

Let us look first at those leaders who have defined themselves based on their competitiveness.
Although at first glance it may appear that a leader must have a competitive edge in order to gain and then maintain a leadership position, I will make two points on this subject.
First, the desire to compete is an inherent part of human nature;
that is, it is not something that needs to be ``instilled'' in young people.
Is there anyone who does not compete in some way or another every single day?
You try to do better than others in your school work or at the office, or you try to do better than yourself in some way, to push yourself.
When societies instill competitiveness in their leaders, it only leads to trouble.
The most \emph{blatant} example in this case is Adolf Hitler, who took competition to the very extreme, trying to prove that his race and his country were superior to all.
We do not, however, need to look that far to find less extreme examples (i.e., Hitler is not the extreme example that disproves the rule).
The recent economic meltdown was caused in no large part by the leaders of American banks and financial institutions who were obsessed with competing for the \emph{almighty} dollar.
Tiger Woods, the ultimate competitor in recent golfing history and in many ways a leader who brought the sport of golf to an entirely new level, destroyed his personal life (and perhaps his career -- still yet to be determined) by his overreaching sense that he could accomplish anything, whether winning majors or sleeping with as women as possible.
His history of competitiveness is well documented; his father pushed him from a very early age to be the ultimate competitor.
It served him well in some respects, but it also proved to be \emph{detrimental} and ultimately quite destructive.

Leaders who value cooperation, on the other hand, have historically been less \emph{prone} to these overreaching, destructive tendencies.
A good cased in point would be Abraham Lincoln.
Now, I am sure at this point you are thinking that Lincoln, who served as President during the Civil War and who refused to compromise with the South or allow \emph{secession}, could not possibly be my model of cooperation.
Think, however, of the way Lincoln structured his Cabinet.
He did not want a group of ``yes men'' who would agree with every word he said, but instead he picked people who were more likely to disagree with his ideas.
And he respected their input, which allowed him to keep the government together in the North during a very \emph{tumultuous} period (to say the least).

My point in choosing the Lincoln example is that competitiveness and conflict may play better to the masses and be more likely to be recorded in the history books, but it was his cooperative nature that allowed him to govern effectively.
Imagine if the CEO of a large company were never able to compromise and insisted that every single thing be done in exactly her way.
Very quickly she would lose the very people that a company needs in order to survive, people with new ideas, people ready to make great advances.
Without the ability to work constructively with those who have conflicting ideas, a leader will never be able to strike deals, reach \emph{consensus}, or keep an enterprise on track.
Even if you are the biggest fish in the pond, it is difficult to force your will on others forever; eventually a bigger fish comes along (or the smaller fish team up against you!).

In the end, it seems most critical for society to instill in young people a sense of cooperation.
In part this is true because we seem to come by our competitive side more naturally, but cooperation is something we struggle to learn (just think of kids on the playground).
And although competitive victory is more showy, more often than not the real details of leadership come down to the ability ti work with other people, to compromise and cooperate.
Getting to be President of United States or the managing director of a corporation might require you to win some battles, but once you are there you will need diplomacy and people-skills.
Those can be difficult to learn, but if you do not have them, you are likely to be a shot-lived leader.


\section{Score 5}
Cooperation, the act of working as a group to achieve a collective goal, is an important value for young children to learn.
Another vital life lesson children can learn is how to be competitive, which is a mindset in which a person feels the need to accomplish more than another person.
Both are necessary to become well rounded individuals, but concerning preparing for a future in government, industry or various other fields, a sense of cooperation is much more important.

Wild not all children are overly competitive in nature, every person has some level of competitive drive inside them.
This is a natural thing and is perfectly normal.
Unfortunately, if this competitive nature is emphasized, the child will have problems relating socially to other children, and \emph{subsequently}, will have issues interacting with adults later in life.
A fierce competitive drive will blind and individual, causing them to not see situations where group effort will be more greatly rewarded then an individual effort.
Take for instance the many teams of people working for NASA.
If the people that make up these teams were all out to prove that they were superior to others, out entire space program would be \emph{jeapordized}.
One needs to look beyond at what will most benefit a broad group of people.
This is where a sense of cooperation in young children is vital.
Cooperation is taught at an early age and must be emphasized throughout life to fully embrace the concept.

In the would of sports a competitive drive is vital; unfortunately, life is not a sports game that simply leads to a winning or losing score.
Life is far more complex than this simple idea and there is no winner or loser \emph{designation} to accompany it.
We all have to work together to come to a conclusion that will assist not just ourselves, but others and future generations.
In every scenario there will be individuals that have brilliant ideas, but those ideas require other people to build upon, perfect and implement.
Take for instance Bill Gates; Bill Gates is responsible for the Microsoft cooperation which he invented in his garage.
His competitive drive assisted in building his idea, but it was the collaborative effort of many people that helped propel his invention into the world known product it is today.
Without the cooperation of others, his genius invention might never have made it out of his garage.

It may be true that an individual can change the world, but only so far as to say that an individual can construct an idea that will inevitably change the world.
Once the idea is formulated, it then takes a team of people working collectively towards a common goal to make sure that the brilliant, life-altering idea makes it to \emph{furtuition}.
Without the cooperation of many, an idea could simply remain as a picture on a drawing board.
It is because of this possibility that instilling a cooperative \emph{demeanor} in children is much more important than developing a competitive attitude.
Competition is a natural thing that will develop with or without encouragement but the same cannot be said for a sense of cooperation.