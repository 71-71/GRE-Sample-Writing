\section{Issue 18}
\paragraph{
Many of the world's lesser-known languages are being lost as fewer and fewer people speak them. The governments of countries in which these languages are spoken should act to prevent such languages from becoming extinct.
}
\subsection{Score 6}


The speaker asserts that governments of countries where lesser-known languages are spoken should intervene to prevent these languages from becoming extinct.
I agree inso far as a country's indigenous and distinct languages should not be abandoned and forgot ten altogether.
At some point, however, I think cultural identity should yield to the more practical considerations of day-to-day life in a global society.


On the one hand, the indigenous language of any geographical region is part-and-parcel of the cultural heritage of the region's natives.
In my observation we humans have a basic psychological need for individual identity, which we define by way of our membership in distinct cultural groups.
A culture defines itself in various ways--by its unique traditions, rituals, mores, attitudes and beliefs, but especially language.
Therefore, when a people's language becomes extinct the result is a diminished sense of pride, dignity, and self- worth.


One need look no further than continental Europe to observe how people cling tenaciously to their distinct languages, despite the fact that there is no practical need for them anymore.


And on the other side of the Atlantic Ocean, the French Canadians stubbornly insist on French as their official language, for the sole purpose of preserving their distinct cultural heritage.


Even where no distinct language exists, people will invent one to gain a sense of cultural identity, as the emergence of the distinct Ebonic cant among today's African Americans aptly illustrates.
In short, people resist language assimilation because of a basic human need to be part of a distinct cultural group.


Another important reason to prevent the extinction of a language is to preserve the distinct ideas that only that particular language can convey.
Certain Native American and Oriental languages, for instance, contain words symbolizing spiritual and other abstract concepts that only these cultures embrace.
Thus, in some cases to lose a language would be to abandon cherished beliefs and ideas that can be conveyed only through language.


On the other hand, in today's high-tech world of satellite communications, global mobility, and especially the Internet, language barriers serve primarily to impede cross-cultural communication, which in turn impedes international commerce and trade.
Moreover, language barriers naturally breed misunderstanding, a certain distrust and, as a result, discord and even war among nations.
Moreover, in my view the extinction of all but a few major languages is inexorable--as supported by the fact that the Internet has adopted English as its official language.
Thus by intervening to preserve a dying language a government might be deploying its resources to fight a losing battle, rather than to combat more pressing social problems--such as hunger, homelessness, disease and ignorance--that plague nearly every society today.


In sum, preserving indigenous languages is, admittedly, a worthy goal; maintaining its own distinct language affords a people a sense of pride, dignity and self-worth.
Moreover, by preserving languages we honor a people's heritage, enhance our understanding of history, and preserve certain ideas that only some languages properly convey.
Nevertheless, the economic and political drawbacks of language barriers outweigh the benefits of preserving a dying language.
In the final analysis, government should devote its time and resources elsewhere, and leave it to the people themselves to take whatever steps are needed to preserve their own distinct languages.
