\section{Issue 25}
\paragraph{
Unfortunately, the media tend to highlight what is sensational at the moment.
Society would be better served if the media reported or focused more fully on events and trends that will ultimately have the most long-term significance.
}
\subsection{Score 6}


The speaker asserts that rather than merely highlighting certain sensational events the media should provide complete coverage of more important events.
While the speaker's assertion has merit from a normative standpoint, in the final analysis I find this assertion indefensible.


Upon first impression the speaker's claim seems quite compelling, for two reasons.
First, without the benefit of a complete, unfiltered, and balanced account of current events, it is impossible to develop an informed and intelligent opinion about important social and political issues and, in turn, to contribute meaningfully to our democratic society, which relies on broad participation in an ongoing debate about such issues to steer a proper course.
The end result of our being a largely uninformed people is that we relegate the most important decisions to a handful of legislators, jurists, and executives who may or may not know what is best for us.


Second, by focusing on the "sensational"--by which I take the speaker to mean comparatively shocking, entertaining, and titillating events which easily catch one's attention-the media appeal to our emotions and baser instincts, rather than to our intellect and reason.
Any observant person could list many examples aptly illustrating the trend in this direction--from trashy talk shows and local news broadcasts to The National Enquixer and People Magazine.
This trend dearly serves to undermine a society's collective sensibilities and renders a society's members more vulnerable to demagoguery; thus we should all abhor and resist the trend.


However, for several reasons I find the media's current trend toward highlights and the sensational to be justifiable.
First, the world is becoming an increasingly eventful place; thus with each passing year it becomes a more onerous task for the media to attempt full news coverage.
Second, we are becoming an increasingly busy society.
The average U.S.
worker spends nearly 60 hours per week at work now; and in most families both spouses work.


Compare this startlingly busy pace to the pace a generation ago, when one bread-winner worked just over 40 hours per week.
We have far less time today for news, so highlights must suffice.
Third, the media does in fact provide full coverage of important events; anyone can find such coverage beyond their newspaper's front page, on daily PBS news programs, and on the Internet.
I would wholeheartedly agree with the speaker if the sensational highlights were all the media were willing or permitted to provide; this scenario would be tantamount to thought control on a mass scale and would serve to undermine our free society.
However, I am aware of no evidence of any trend in this direction.
To the contrary, in my observation the media are informing us more fully than ever before; we just need to seek out that information.


On balance, then, the speaker's claim is not defensible.
In the final analysis the media serves its proper function by merely providing what we in a free society demand.
Thus any argument about how the media should or should not behave--regardless of its merits from a normative standpoint begs the question.
