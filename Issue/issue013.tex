\section{Issue 13}
\paragraph{
A nation should require all its students to study the same national curriculum until they enter college rather than allow schools in different parts of the nation to determine which academic courses to offer.
}
\subsection{Score 6}


The speaker would prefer a national curriculum for all children up until college instead of allowing schools in different regions the freedom to decide on their own curricula.
I agree insofar as some common core curriculum would serve useful purposes for any nation.
At the same time, however, individual states and communities should have some freedom to augment any such curriculum as they see fit; otherwise, a nation's educational system might defeat its own purposes in the long tenn.


A national core curriculum would be beneficial to a nation in a number of respects.
First of all, by providing all children with fundamental skills and knowledge, a common core curriculum would help ensure that our children grow up to become reasonably informed, productive members of society.
In addition, a common core curriculum would provide a predictable foundation upon which college administrators and faculty could more easily build curricula and select course materials for freshmen that are neither below nor above their level of educational experience.
Finally, a core curriculum would ensure that all school-children are taught core values upon which any democratic society depends to thrive, and even survive--values such as tolerance of others with different viewpoints, and respect for others.


However, a common curriculum that is also an exdusive one would pose certain problems, which might outweigh the benefits, noted above.
First of all, on what basis would certain course work be included or excluded, and who would be the final decision- maker? In all likelihood these decisions would be in the hands of federal legislators and regulators, who are likely to have their own quirky notions of what should and should not be taught to children--notions that may or may not reflect those of most communities, schools, or parents.


Besides, government officials are notoriously susceptible to influence-peddling by lobbyists who do not have the best interests of society's children in mind.


Secondly, an official, federally sanctioned curriculum would facilitate the dissemination of propaganda and other dogma which because of its biased and one-sided nature undermines the very purpose of true education: to enlighten.
I can easily foresee the banning of certain text books, programs, and websites which provide information and perspectives that the government might wish to suppress--as some sort of threat to its authority and power.


Although this scenario might seem far-fetched, these sorts of concerns are being raised already at the state level.


Thirdly, the inflexible nature of a uniform national curriculum would preclude the inclusion of programs, courses, and materials that are primarily of regional or local significance.
For example, California requires children at certain grade levels to learn about the history of particular ethnic groups who make up the state's diverse population.
A national curriculum might not allow for this feature, and California's youngsters would be worse off as a result of their ignorance about the traditions, values, and cultural contributions of all the people whose citizenship they share.


Finally, it seems to me that imposing a uniform national curriculum would serve to undermine the authority of parents over their own children, to even a greater extent than uniform state laws currently do.
Admittedly, laws requiring parents to ensure that their children receive an education that meets certain minimum standards are well-justified, for the reasons mentioned earlier.
However, when such standards are imposed by the state rather than at the community level parents are left with far less power to participate meaningfully in the decision-making process.
This problem would only be exacerbated were these decisions left exclusively to federal regulators.


In the final analysis, homogenization of elementary and secondary education would amount to a double-edged sword.
While it would serve as an insurance policy against a future populated with illiterates and ignoramuses, at the same time it might serve to obliterate cultural diversity and tradition.
The optimal federal approach, in my view, is a balanced one that imposes a basic curriculum yet leaves the rest up to each state--or better yet, to each community.
