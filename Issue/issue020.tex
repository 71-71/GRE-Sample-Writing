\section{Issue 20}
\paragraph{
Most cultures encourage individuals to sacrifice a large part of their own personalities in order to be like other people.
Thus, most people are afraid to think or behave differently because they do not want to be excluded.
}
\subsection{Score 6}


The speaker claims that most cultures encourage conformity at the expense of individuality, and as a result most people conform for fear of being excluded.
While I find the second prong of this dual claim well supported overall by empirical evidence, I take exception with the first prong; aside from the cultures created by certain oppressive political regimes, no culture need "encourage" its members to conform to prevailing ways of thought and behavior; in fact, all the evidence shows that cultures attempt to do just the opposite.


As a threshold matter, it is necessary to distinguish between conformity that an oppressive ruling state imposes on its own culture and conformity in a free democratic society.
In the former case, people are not only encouraged but actually coerced into suppressing individual personality; and indeed these people are afraid to think and behave differently--but not for fear of being excluded but rather for fear of punishment and persecution by the state.
The modern Communist and Fascist regimes are fitting examples.
With respect to free democratic societies, it might be tempting to dismiss the speaker's dual claim out of hand.
After all, true democratic states are predicated on individual freedoms---of choice, speech, expression, religion, and so forth.
Ostensibly, these freedoms serve to promote individuality, even non-conformity, in our personas, our lifestyles, and our opinions and attitudes.


Yet, one look at any democratic society reveals a high degree of conformity among its members.
Every society has its own bundle of values, customs, and mores which most of its members share.
Admittedly, within any culture springs up various subcultures which try to distinguish themselves by their own distinct values, customs, and mores.
In the U.S., for instance, African-Americans have developed a distinct dialect, known as Ebonics, and a distinct body language and attitude which affords them a strong sub-cultural identity of their own.
Yet, the undeniable fact is that humans, given the actual freedom to either conform or not conform, choose to think and behave in ways similar to most people in their social group---however we define that group.


Nor is there much empirical evidence of any cultural agenda, either overt or covert, to encourage conformity in thought and behavior among the members of any culture.
To the contrary, the predominant message in most cultures is that people should cultivate their individuality.
Consider, for example, the enduring and nearly ubiquitous icon of the ragged individualist, who charts his or her own course, bucks the trend, and achieves notoriety through individual creativity, imagination, invention, or entrepreneurship.
Even our systems of higher education seem to encourage individualism by promoting and cultivating critical and independent thought among its students.


Yet, all the support for forging one's one unique persona, career, lifestyle, opinions, and even belief system, turns out to be hype.
In the final analysis, most people choose to conform.


And understandably so; after all, it is human nature to distrust, and even shun, others who are too different from us.
Thus to embrace rugged individualism is to risk becoming an outcast, the natural consequence of which is to lLmit one's socioeconomic and career opportunities.
This prospect suffices to quell our yearning to be different; thus the speaker is correct that most of us resign ourselves to conformity for fear of being left behind by our peers.
Admittedly, few cultures are without rugged individualists----the exceptional artists, inventors, explorers, social reformers, and entrepreneurs who embrace their autonomy of thought and behavior, then test their limits.
And paradoxically, it is the achievements of these notable non-conformists that are responsible for most cultural evolution and progress.
Yet such notables are few and far between in what is otherwise a world of insecure, even fearful, cultural conformists.


To sum up, the speaker is correct that most people choose to conform rather than behave and think in ways that run contrary to their culture's norms, and that fear of being exduded lies at the heart of this choice.
Yet, no culture need encourage conformity; most humans recognize that there is safety of numbers, and as a result freely choose conformity over the risks, and potential rewards, of non-conformity.
