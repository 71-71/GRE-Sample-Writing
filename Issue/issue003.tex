\section{Issue 3}

\paragraph{Some people believe that corporations have a responsibility to promote the well-being of the socities and environments in which they operate.
Others believe that the only responsibility of corporations, provided they operate within the law, is to make as much money as possible.}

\subsection{Score 6}
It is not uncommon for some to argue that, in the world in which we live, corporations have a responsibility to society and to the environment in which they operate.
\emph{Proponents} of this view would argue that major environmental catastrophes (e.g., the oil spill in the Gulf) are key examples of the damage that can be wrought when corporations are allowed to operate unchecked.
Yet within that very statement lies a contradiction that undermines this kind of thinking--it is necessary for outside forces to check the behavior of corporations, because we do not expect corporations to bevave in such manner.
In fact, the expectation is simply that corporations will follow the law, and in the course of doing so, engage in every possible tactic to their advantage in the pursuit of more and greater profit.
To expect otherwise from corporations is to fail to understand their purpose and their very structure.

The corporation arose as a model of business in which captital could be raised through the contributions of stockholders: investors purchase shares in a company, and their money is then used as the operating capital for the company.
Shareholders buy stock not because they are hoping to better make the world a better place or because they have a desire to improve the quality of life but because they expect to see a return in their investment in this company.
The company may itself have generally \emph{altruistic} goals (perhaps it is a think tank that advises the government on how to improve relations with the Middle East, or perhaps it is a company built around finding alternative forms of energy), but the immediate expectation of the investor is that himself will see dividends, or profits, from the investment he has made.
This is even more true in the case of companies that are purely profit driven and which do not have goals that are particularly directed toward social environment--a description that applies to the vast majority of corporations.

Is it a bad thing to have a corporation negatively affact the environment (and by extension, its inhabitants)?
To pump noxious fumes into the atmosphere as a by-product of its manufacturing process?
Of course, and this is why agencies such as the EPA were established and why governments--federal, state and local--are expected to monitor such companies.
Any and all corporations should be expected to temper their input pursuit of profit with the necessity of following those safeguards that have been legislated as protections.
But the assumption that corporations have a inherent obligation or responsibility to go above and beyond that to actively PROMOTE the environment and the well-being of society is absurd.

Engaging in practices to adhere to legal expectations to protect society and the environment is costly to corporations.
If the very purpose of a corporation is to generate profits, and the obligation to adhere to safety expectations established by law cuts into those profits, then to expect corporations to embrace such practices beyonf what is required to presume that they willingly engage in an inherently self-destructive process: the unnecessary lowering of profits.
This is \emph{antithetical} to the very concept of the corporation.
Treehuggers everywhere should be pleased that environmental protections exist, but to expect corporations to ``make the world a better place'' is to embrace altruism to the point that it becomes \emph{delusion}.

This is not to say that we should reject efforts to hold corporations accountable.
In fact, the opposite is true--we should be \emph{vigilant} with the business world and maintain our expectations that corporations do not make their profits at the EXPENSE of the well-being of society.
But that role must be fulfilled by a watchdog, not the corporation itself, and those expectations mush be imposed UPON the corporations, not expected FROM them.