\section{Issue 17}
\paragraph{
All nations should help support the development of a global university designed to engage students in the process of solving the world's most persistent social problems.
}
\subsection{Score 6}


I agree that it would serve the interests of all nations to establish a global university for the purpose of solving the world's most persistent social problems.
Nevertheless, such a university poses certain risks which all participating nations must be careful to minimize--or risk defeating the university's purpose.


One compelling argument in favor of a global university has to do with the fact that its faculty and students would bring diverse cultural and educational perspectives to the problems they seek to solve.
It seems to me that nations can only benefit from a global university where students learn ways in which other nations address certain soda] problems-successfully or not.


It might be tempting to think that an overly diversified academic community would impede communication among students and faculty.
However, in my view any such concerns are unwarranted, especially considering the growing awareness of other peoples and cultures which the mass media, and especially the Internet, have created.
Moreover, many basic principles used to solve enduring social problems know no national boundaries; thus a useful insight or discovery can come from a researcher or student from any nation.


Another compelling argument for a global university involves the increasingly global nature of certain problems.
Consider, for instance, the depletion of atmospheric ozone, which has wanned the Earth to the point that it threatens the very survival of the human species.
Also, we are now learning that dear-cutting the world's rainforests can set into motion a chain of animal extinction that threatens the delicate balance upon which all animals--including humans--depend.
Also consider that a financial crisis---or a political crisis or natural disaster in one country can spell trouble for foreign companies, many of which are now multinational in that they rely on the labor forces, equipment, and raw materials of other nations.


Environmental, economic, and political problems such as these all carry grave social consequences--increased crime, unemployment, insurrection, hunger, and so forth.
Solving these problems requires global cooperation--which a global university can facilitate.


Notwithstanding the foregoing reasons why a global university would help solve many of our most pressing social problems, the establishment of such a university poses certain problems of its own which must be addressed in order that the university can achieve its objectives.
First, participant nations would need to overcome a myriad of administrative and political impediments.
All nations would need to agree on which problems demand the university's attention and resources, which areas of academic research are worthwhile, as well as agreeing on policies and procedures for making, enforcing, and amending these decisions.


Query whether a functional global university is politically feasible, given that sovereign nations naturally wish to advance their own agendas.


A second problem inherent in establishing a global university involves the risk that certain intellectual and research avenues would become officially sanctioned while others of equal or greater potential value would be discouraged, or perhaps even proscribed.
A telling example of the inherent danger of setting and enforcing official research priorities involves the Soviet government's attempts during the 1920s to not only control the direction and the goals of its scientists' research but also to distort the outcome of that research---ostensibly for the greatest good of the greatest number of people.
Not surprisingly, during this time period no significant scientific advances occurred under the auspices of the Soviet government.
The Soviet lesson provides an important caveat to administrators of a global university: Significant progress in solving pressing social problems requires an open mind to all sound ideas, approaches, and theories---krespective of the ideologies of their proponents.


A final problem with a global university is that the world's preeminent intellectual talent might be drawn to the sorts of problems to which the university is charged with solving, while parochial social problem go unsolved.
While this is not reason enough not to establish a global university, it nevertheless is a concern that university administrators and participant nations must be aware of in allocating resources and intellectual talent.


To sum up, given the increasingly global nature or the world's social problems, and the escalating costs of addressing these problems, a global university makes good sense.
And, since all nations would have a common interest in seeing this endeavor succeed, my intuition is that participating nations would be able to overcome whatever procedural and political obstacles that might stand in the way of success.
As long as each nation is careful not to neglect its own unique social problems, and as long as the university's administrators are careful to remain open-minded about the legitimacy and potential value of various avenues of intellectual inquiry and research, a global university might go along way toward solving many of the world's pressing social problems.
