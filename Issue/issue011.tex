\section{Issue 11}

\paragraph{We can usually learn much more from people whose views we share than from people whose views contradict our own.
Disagreement can cause stress and inhibit learning.}

\subsection{Score 6}
Do we learn more from people whose ideas we share in common than from those whose ideas contradict ours?
The speaker daims so, for the reason that disagreement can cause stress and inhibit learning.
I concede that undue discord can impede learning.
Otherwise, in my view we learn far more from discourse and debate with those whose ideas we oppose than from people whose ideas are in accord with our own.

Admittedly, under some circumstances disagreement with others can be counterproductive to learning.
For supporting examples one need look no further than a television set.
On today's typical television or radio talk show, disagreement usually manifests itself in meaningless
rhetorical bouts and shouting matches, during which opponents vie to have their own message heard, but have little interest either in finding common ground with or in acknowledging the merits of the opponent's viewpoint. 
Understandably, neither the combatants nor the viewers learn anything meaningful.
In fact, these battles only serve to reinforce the predispositions and biases of all concerned.
The end result is that learning is impeded.

Disagreement can also inhibit learning when two opponents disagree on fundamental assumptions needed for meaningful discourse and debate.
For example, a student of paleontology learns little about the evolution of an animal species under current study by debating with an individual whose religious belief system precludes the possibility of evolution
to begin with.
And, economics and finance students learn little about the dynamics of a laissez-faire system by debating with a socialist whose view is that a centralized power should
control all economic activity.

Aside from the foregoing two provisos, however, I fundamentally disagree with the speaker's claim.
Assuming common ground between two rational and reasonable opponents willing to debate on intellectual merits, both opponents stand to gain much from that debate.
Indeed it is primarily through such debate that human knowledge advances, whether at the personal, community, or global level.

At the personal level, by listening to their parents' rationale for their seemingly oppressive rules and policies teenagers can learn how certain behaviors naturally carry certain undesirable consequences.
At the same time, by listening to their teenagers concerns about autonomy and about peer pressures parents can learn the valuable lesson that effective parenting and control are two different things.
At the community level, through dispassionate dialogue an environmental activist can come to understand the legitimate economic concerns of those whose jobs depend on the continued profitable operation of a factory.
Conversely, the latter might stand to learn much about the potential public health price to be paid by ensuring job growth and a low unemployment rate.
Finally, at the global level, two nations with opposing political or economic interests can reach mutually beneficial agreements by striving to understand the other's legitimate concerns for its national security, its political sovereignty, the stability of its economy and currency, and so forth.

In sum, unless two opponents in a debate are each willing to play on the same field and by the same rules, I concede that disagreement can impede learning.
Otherwise, reasoned discourse and debate between people with opposing viewpoints is the very foundation upon which human knowledge advances.
Accordingly, on balance the speaker is fundamentally correct.