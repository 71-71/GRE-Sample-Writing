\section{Issue 15}
\paragraph{
It is often necessary, even desirable, for political leaders to withhold information from the public.
}
\subsection{Score 6}


I agree with the speaker that it is sometimes necessary, and even desirable, for political leaders to withhold information from the public.
A contrary view would reveal a naivetd about the inherent nature of public politics, and about the sorts of compromises on the part of well-intentioned political leaders necessary in order to further the public's ultmaate interests.


Nevertheless, we must not allow our political leaders undue freedom to with-hold information, otherwise, we risk sanctioning demagoguery and undermining the philosophical underpinnings of any democratic society.


One reason for my fundamental agreement with the speaker is that in order to gain the opportunity for effective public leadership, a would-be leader must fzrst gain and maintain political power.
In the game of politics, complete forthrightness is a sign of vulnerability and naivete, neither of which earn a politician respect among his or her opponents, and which those opponents will use to every advantage to defeat the politician.
In my observation some measure of pandering to the electorate is necessary to gain and maintain political leadership.


For example, were all politicians to fully disclose every personal foibles, character flaw, and detail concerning personal life, few honest politicians would ever by elected.
While this view might seem cynical, personal scandals have in fact proven the undoing of many a political career; thus I think this view is realistic.


Another reason why I essentially agree with the speaker is that fully disclosing to the public certain types of information would threaten public safety and perhaps even national security.


For example, if the President were to disclose the government's strategies for thwarting specific plans of an international terrorist or a drug trafficker, those strategies would surely fail, and the public's health and safety would be compromised as a result.
Withholding information might also be necessary to avoid public panic.
While such cases are rare, they do occur occasionally.
For example, during the first few hours of the new millennium the U.S.


Pentagon's missile defense system experienced a Y2K- related malfunction.
This fact was withheld from the public until later in the day, once the problem had been solved; and legitimately so, since immediate disclosure would have served no useful purpose and might even have resulted in mass hysteria.


Having recognized that withholding informarion from the public is often necessary to serve the interests of that public, legitimate political leadership nevertheless requires forthrightness with the citizenry as to the leader's motives and agenda.
History informs us that would-be leaders who lack such forthrightness are the same ones who seize and maintain power either by brute force or by demagoguery--that is, by deceiving and manipulating the citizenry.


Paragons such as Genghis Khan and Hitler, respectively, come immediately to mind.
Any democratic society should of course abhor demagoguery, which operates against the democratic principle of government by the people.
Consider also less egregious examples, such as President Nixon's withholding of information about his active role in the Watergate cover-up.
His behavior demonstrated a concern for self- interest above the broader interests of the democratic system that granted his political authority in the first place.


In sum, the game of politics calls for a certain amount of disingenuousness and lack of forthrightness that we might otherwise characterize as dishonesty.
And such behavior is a necessary means to the final objective of effective political leadership.
Nevertheless, in any democracy a leader who relies chiefly on deception and secrecy to preserve that leadership, to advance a private agenda, or to conceal selfish motives, betrays the democracy-and ends up forfeiting the polirical game.
