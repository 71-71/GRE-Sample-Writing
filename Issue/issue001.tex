\section{Issue 1}
\paragraph{The best way to teach is to praise positive actions and ignore negative ones.}

\subsection{Score 6}
The recommendation presents a view that I would agree is successful most of the time, but one that I cannot fully support due to the ``all or nothing'' impression it gives.

Certainly as an educator I agree fully that the best way to \emph{elicit} positive response from students is to make use of students' positive energy and then encourage actions that you would like to see repeated.
It is human nature that we all want to be accepted and achieve on some level, and when people in authority provide feedback that we have done something well, the drive to repeat the action that was praised is bound to be particularly strong.

This blanket statement would obviously pay dividends in situations in which a teacher desires to have students repeat particular behaviors.
For example, if an educator is attempting to teach students proper classroom \emph{etiquette}, it would be appropriate to openly praise a student who raises his or her hand when wishing to speak or address the class.
In such cases, the teacher may also help shape positive behaviors by ignoring a student who is trying to interject without approval from the teacher.
In fact, the decision to ignore students who are exhibiting inappropriate behaviors of this type could work very well in this situation, as the stakes are not very high and the intended outcome can likely be achieved by such a method.
However, it is important to note here that this tactic would only be effective in such a ``low-stakes'' situation, as when a student speaks without raising her hand first.
As we will discuss below, ignoring a student who hits another student, or engages in more serious misbehavior, would not be effective or \emph{prudent}.

To expand on this point, it is important for teachers to be careful when working with a second half of this statement, only ignoring negative actions that are not serious.
Take for instance a student who is misbehaving just by chatting with a fellow classmate.
This student might not be presenting much of a problem and may be simply seeking attention.
Ignoring the student might, in fact, be the best solution.
Now assume the negative action is the improper administering of chemicals in a science experiment or the bullying of a fellow student.
To ignore these negative actions would be absurd and \emph{negligent}.
Now you are allowing a problem to persist, one that could potentially lead to much bigger and more dangerous issues.
In a more serious situation addressing the negative actions quickly and properly could stop the problem in its tracks.
It is for reasons like this that I do not \emph{advocate} the idea that a teacher can be successful by simply ignoring negative actions.

I do, however, greatly support the idea that the central focus of teaching should be to build on and encourage positive actions.
However, the author's \emph{encompasing} statement leaves too many negative possibilities for the classroom.
Perhaps a better way to phrase this statement would be to say, ``The best way to teach is to praise positive actions and ignore negative ones that are not \emph{debilitating} to class efficiency or the safety of any individual''.

Thus, in the original statement, there are indeed some good intentions, and there could be a lot of \emph{merit} in adopting its basic principles.
Data proves that positive support can substantially increase motivation and desire in students and contribute to positive achievements.
In fact, most studies of teaching efficiency indicate that praising positive actions and ignoring negative ones can create a more stable and efficient classroom.
It needs to be stressed, however, that this tool is only effective at certain levels of misbehavior.
As mentioned above, when a behaviors is \emph{precipitated} by feelings of revenge, power or total self-worthlessness, this methodology will likely not work.
It is likely to be very successful, however, when the drive behind the misbehavior is simple attention seeking.
In many of these instances, if the teacher demonstrates clearly that inappropriate behavior does not result in the gaining of attention, students are more likely to seek attention by behaving properly.
Should the student choose this path, then the ignoring has worked and when the positive behavior is exhibited, then the teacher can utilize the first part of the theory and support or praise this behavior.
Now it is much more likely to be repeated.
If the student does not choose this path and instead elects to raise the actions to a higher level that presents a more serious issue, then ignorance alone cannot work and other methods must be employed.

In conclusion, one can appreciate the \emph{credo} expressed in this instance, but surely we all can see the potential error of following it through to the extreme.





\subsection{Score 5}
I partially agree with the statement ``The best way to teach is to praise positive actions and ignore negative onesd''.
Children should be rewarded when they perform well;
however, they should bot be ignored for performing sub-optimally.
For purposes of this essay, the term ``actions'' is defined as behaviors within the classroom.

Utilizing positive reinforcements, such as \emph{tangible} rewards, can be a good method to teach children.
If the teacher praises children for actions that are desirable, then the children are more likely to repeat those actions.
For example, a student who completes an assignment on time and does a good job is likely to want to do a good job on the next assignment if he gets positive feedback.
Likewise, the children who are not currently engaging in the desirable actions may be more inclined to do so in order to receive the positive reinforcement.

\emph{Conversely}, children should not be ignored for negative actions.
If a child is not exhibiting appropriate behavior in the classroom, then it is the teacher's responsibility to encourage the child to perform optimally.
Ignoring something doesn't make it go away, actions and consequences do.
A student who is being disruptive in class will continue to be disruptive unless the teacher does something about it.
However, the teacher's actions need to be appropriate.

Before the teacher attempts to modify a child's behavior, the teacher needs to try and identify the reason behind the behavior.
For instance, children who leave their seats often, stare into space, or \emph{call out of turn} may be initially viewed as having poor behavior.
However, the teacher may suspect that the child has an attentional problem, and request that child be tested.
If the child does not have an attentional problem, then the teacher can work with a related service, such as occupational therapy, to alter the classroom environment in order to cater to the needs of the child.
For instance, the teacher could remove some of the stimulating bulliten board displays to make the room more calming to the child.
If the child becomes more attentive in class then the teacher was able to assist the child without \emph{scorning} them or ignoring them.
The teacher met the needs of the child and created and environment to enable the child and created an environment to enable the child to optimally perform in the educational setting.

On the other hand, if the child is tested and does not have any areas of concern that may be impacting the educational performance in the classroom, then the negative behavior may strictly be due to \emph{defiance}.
In such a case, the teacher still should bot ignore the child, because the negative actions may \emph{hinder} the learning opportunity for the remaining children in the class.
As a result, a child who is being disruptive to the learning process of the class should be set apart from the class so that they do not receive the positive reinforcement of peer attention.

The teacher should not ignore the student who is misbehaving, but that does not mean that the teacher just needs to punish.
It is better to address the child privately and make sure the child is aware of the negative actions.
Once the child is aware, then the teacher should once again try to determine the reason why the child is behaving in a negative manner.
Perhaps the child's parents are in the middle of a divorce and the child is outwardly expressing his frustration in the classroom.
Or the academic content of the class may not be challenging enough for the child and so he is misbehaving out of boredom.
Whatever the reason behind the behavior, the hey factor is that the teacher works with the child to try and identify it,
Simply punishing or ignoring the child would not solve the problem, whereas working to create a plan for success in the classroom would.
Likewise, rather than punishing and defeating the child, the teacher is working with and \emph{empowering} the child; a much more positive outcome to the situation.