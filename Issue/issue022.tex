\section{Issue 22}
\paragraph{
Anyone can make things bigger and more complex.
What requires real effort and courage is to move in the opposite direction---in other words, to make things as simple as possible.
Lending even more credence to the statement is the so-called big government
}
\subsection{Score 6}


Whether making things simple requires greater effort and courage than making them bigger and more complex depends on the sort of effort and courage.
Indisputably, the many complex technological marvels that are part-and-parcel of our Lives today are the result of the extraordinary cumulative efforts of our engineers, entrepreneurs, and others.
And, such achievements always call for the courage to risk failing in a large way.
Yet, humans seem naturally driven to make things bigger and more complex; thus refraining from doing so, or reversing this natural process, takes considerable effort and courage of a different sort, as discussed below.


The statement brings immediately to mind the ever-growing and increasingly complex digital world.
Today's high-tech firms seem compelled to boldly go to whatever effort is required to devise increasingly complex products, for the ostensible purpose of staying ahead of their competitors.
Yet, the sort of effort and courage to which the statement refers is a different one--bred of vision, imagination, and a willingness to forego near term profits for the prospect of making lasting contributions.
Surely, a number of entrepreneurs and engineers today are mustering that courage, and are making the effort to create far simpler, yet more elegant, technologies and applications, which will truly make our lives simpler in sharp contrast to what computer technology has delivered to us so far.
phenomenon.
Human societies have a natural tendency to create unwieldy bureaucracies, a fitting example of which is the U.S.
tax-law system.
The Intemal Revenue Code and its accompanying Treasury Regulations have grown so voluminous and complex that many certified accountants and tax attorneys admit that they cannot begin to understand it all.


Admittedly, this system has grown only through considerable effort on the part of all three branches of the federal government, not to mention the efforts of many special interest groups.
Yet, therein lies the statement's credibility.
It requires great effort and courage on the part of a legislator to risk alienating special interest groups, thereby risking reelection prospects, by standing on principle for a simpler tax system that is less costly to administer and better serves the interests of most taxpayers.


Adding further credibility to the statement is the tendency of most people to complicate their personal lives--a tendency that seems especially strong in today's age of technology and consumerism.
The greater our mobility, the greater our number of destinations each day; the more time-saving gadgets we use, the more activities we try to pack into our day; and with readier access to information we try to assimilate more of it each day.
I am hard-pressed to think of one person who has ever exclaimed to me how much effort and courage it has taken to complicate his or her life in these respects.
In contrast, a certain self-restraint and courage of conviction are both required to eschew modern conveniences, to simplify one'sdaily schedule, and to establish and adhere to a simple plan for the use of one's time and money.


In sum, whether we are building computer networks, government agencies, or personal lifestyles, great effort and courage are required to make things simple, or to keep them that way.
Moreover, because humans na~traUy tend to make things big and complex, it arguably requires more effort and courage to move in the opposite direction.
In the final analysis, making things simple---or keeping them that way--takes a brand of effort born of reflection and restraint rather than sheer exertion, and a courage character and conviction rather than unbridled ambition.
