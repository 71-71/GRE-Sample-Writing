\section{Argument 1}
\paragraph{The following appeared in an article written by Dr. Karp, an anthropologist.}
\subparagraph{Twenty years ago, Dr. Field, a noted anthropologist, visited the island of Tertia and concluded from his observations that children in Tertia were reared by an entire village rather than by their own biological parents.
However, my recent interviews with children living in the group of islands that includes Tertia show that these children spend much more time talking about their biological parents than about other adults in the village.
This research of mine proves that Dr. Field's conclusion about Tertian village culture is invalid and thus that the observation-centered approach to studying cultures is invalid as well.
The interview-centered method that my team of graduate students is currently using Tertia will establish a much more accurate understanding of child-rearing traditions there and in other island cultures.}


\subsection{Score 6}
It might seem logical, at first glance, to agree with the argument in Dr. Karp's article that children in Tertia actually are raised by their biological parents (and perhaps even, by implication that an observation-centered approach to anthropological study is not as valid as an interview-centered one).
However, in order to fully evaluate this argument, we need to have a significant amount of additional evidence.
The argument could end up being much weaker than it seems, or it might actually quite valid.
In order to make that determination, we need to know more then analyze what we learn.

The first piece of evidence that we would need in order to evaluate Dr. Karp's claims is information about whether or not Tertia and the surrounding island group have changed significantly in the past 20 years.
Dr. Field conducted his observational study 20 years ago, and it is possible that Tertia has changed significantly since then.
For example, if we had evidence that in the intervening years Westerners had settled on the island and they introduced a more typical Western-style family structure, it would certainly weaken Dr. Karp's argument.
In that case, the original study could have been accurate, and Dr. Karp's study could be correct, as well, though his conclusion that Dr. Field's method is ineffective would be seriously weakened.

Another piece of evidence that might help us evaluate this claim involve the exact locations where Dr. Karp's interviews took place.
According to this article, Dr. Karp and his graduate students conducted interviews of ``children living in the group of islands that includes Tertia.''
If we were to learn that they never interviews a single Tertian child, it would significantly weaken the conclusion.
It could tun out to be the case, for example, that children on Tertia are raised \emph{communally}, whereas children on other islands nearby are raised by their biological parents.

In order to fully evaluate this article, we would also need to learn more about the interview questions that Dr. Karp's team used.
What exactly did they ask?
We don't know, nor do we know what the children's responses actually were.
What did they say about their biological parents?
The mere fact that they speak more frequently about their biological parents than they do about other adults does not mean that they are raised by their parents.
It would significantly \emph{undermine} Dr. Karp's argument if it turned out that the children said thing like how much they missed their parents or how their parent had left them in a communal environment.
Without knowing WHAT the children said, it is hard to accept Dr. Karp's conclusion.

It is slightly more difficult to discuss the evidence we might need in order to evaluate the more interesting claims in Dr. Karp's article, namely this extension of the results of his study to a conclusion that interviewed-centered methods are \emph{inherently} more valid than observational-centered approaches.
In order to fully evaluate this claim, in fact, we would need to look at many more examples of interview-based and observation-based anthropological studies and we would also need to look into different study designs.
Perhaps Dr. Field did not conduct an effective observation study, but other observational approaches could be effective.
In order to make such \emph{grandiose} claims, Dr. Karp really needs a lot of additional evidence (ideally a meta-analysis of hundred of anthropological studies).

Clearly, then, we need to have additional evidence in order to get a more complete understanding of the strengths and weaknesses of Dr. Karp's article.
We need to know about Tertia and the surrounding islands, whether or not they have changed over the past 20 years.
We also need to know about study design (Dr. Karp's and Dr. Field's).
And we really need a lot more information if we want to extend the results of a study about one island culture to all anthropological fieldwork.



\subsection{Score 5}
There seems to be an \emph{abundance} of evidence that, if we were to examine it closely, might make us reconsider Dr. Karp's argument here.
If we look first at the evidence that might weaken this argument, we can see a lot of problems with Dr. Karp's article.
It would certainly weaken the argument if we were to discover that Dr. Karp and his students did not actually conduct any of their interviews on the island of Tertia itself.
Looking closely at the article, we see that Dr. Karp claims that the island group that includes Tertia.
There is no evidence that they interviewed Tertian children.
It would definitely weaken the argument if we were to learn that they interviewed children only on islands close to Tertia.
Those islands may or may not have similar child-rearing traditions, and geographic proximity does not guarantee \emph{societal} similarity.

Another piece of evidence that would weaken the argument could come transcripts of the interviews themselves.
Dr. Karp's article makes the claim that the children ``spend much more time talking about their biological parents than about other adults,'' but he gives no indication of what exactly they say about their biological parents.
After all, the children may be talking about how they never see their parents.

One more important piece of evidence that might \emph{undermine} the argument Dr. Karp is making in this article.
He admits that twenty years have passed since Dr. Field's study was conducted, but he does not provide evidence that proves child-rearing techniques have not changed significantly in that time.
Any number of factors could have led to a significant shift in how children are raised.
Influences form other cultures, significant catastrophic evens, or a change in government structures could have led to a change in family dynamics.
Any evidence of such changes would clearly undermine Dr. Karp's argument.

If we went looking for evidence that could strengthen the argument, we might also find something interesting.
Obviously, some of the evidence above might strengthen the argument if they were NOT as discussed above (e.g., if there were evidence that the Tertian islands have NOT changed since Dr. Field's study of if there were transcripts that showed the children spoke about how much they loved living with their biological parents).
However, if we discovered that there are numerous interview-based studies that confirm Dr. Karp's findings, it would go a long way toward \emph{bolstering} his claim about Tertian child-rearing AND his claim about interview-centered studies being more effective.
Another piece of evidence that would strengthen Dr. Karp's argument is undermining Dr. Field's original study.
Maybe Dr. Field was sloppy, for example.

Dr. Karp's article, then, ends up looking something of an empty shell.
Depending on the evidence we find to fill it out, we may discover that it is quite convincing, or we could determine that he is full of hot air.