\section{Argument 2}

\paragraph{The following is taken from a memo from the advertising director of the Super Screen Movie Production Company.}
\subparagraph{According to a recent report from our marketing department, during the past year, fewer people attended Super Screen-produced movies than in any other year.
And yet the percentage of positive reviews by movie reviewer about specific Super Screen movies actually increased during the past year.
Clearly, the contents of these reviews are not reaching enough of our prospective viewers.
Thus, the problem lies not with the quality of our movies but with the public's lack of awareness that movies of good quality are available.
Super Screen should therefore allocate a greater share of its budget next year to reaching the public through advertising.}

\subsection{Score 6}
The advertising director of the Super Screen Movie production company believes that increasing the amount of advertising the company will increase the amount of people attending Super Screen produced movies.
He believes this because during the past year fewer people than ever before attended Super Screen produced movies, yet the percentage of positive reviews increased over the past year.
\emph{Ostensibly}, the extra advertising would \emph{tout} the good reviews written about Super Screen movies.
Before this plan is implemented , however, Super Screen needs to address some questions about its possible \emph{flaws}.

First of all, the company needs to ask what the actual number of people attending its movies as compared to the movies of other production companies is.
The number of people going to movies may have been in universal decline.
If this is the case and more people are going to see Super Screen Movies than the movies of any other production company, advertising about how fun it is to go to the movie theater may do more to boost Super Screen viewership than advertising promoting its own good reviews.

Secondly, the company needs to ask what the actual original number of positive reviews was.
If Super Screen movies received 1\% positive reviews last year and this year they received 2\% positive reviews, getting the message to viewers is not going to increase Super Screen attendence.
Making better movies would be much more likely to increase attendence rates.

Finally, Sper Screen needs to ask what the relationship is between its viewers and the movie reviewers cited in the meno.
Using a survey distributed to its target audience, Sper Screen could determine if movie reviews have an effect on their audience's decision to go to see a movie, whether movie reviewers tended to have the same taste as the target audience.
Super Screen also needs to consider how its movie choices have affected the separate movie reviewer and audience populations.
If the studio has switched from making mega-blockbuster action movies to more nuanced dramas, the general public may be less willing to go see their movies even though movie critic prerfer the dramas to the action movies.

Finally the studio must ask whether the percentage of positive reviews is really a relevant way to measure the potential impact of move reviews.
There are dozens of movie reviewers but when deciding whether to not to go to a movie, the general public will usually pick from among the 10 most popular movie reviews.
These are the reviews that will impress th public if they are included in advertising.
If the most popular movie reviewers disliked Super Screen movies that a larger number of small time film bloggers reviewed positively, Super Screen needs to think of a new advertising strategy.

In conclusion, there are many questions Super Screen needs to answer before using this advertising director's plan.
They need to look carefully at actual numbers, both of viewsip and of positive reviews.
They also need to identify the relationship and their target audience has with movie reviewers and determine how their target audience feels about their movies.
Finally they need to take a nuanced look at the movie reviews that they use in their advertising.


\subsection{Score 5}
While the advertising director clearly aims at \emph{relitalizing} his production company and ensuring that the public is well informed about the movies which are available, there are several basic flaws to this argument.
There remain some questions that need answering before any steps can be taken with regard to advertising strategies for the Super Screen Movie Production Company.

First among these questions is this: were ticket sales of the entire movie industry down?
This is an essential question because it helps to pinpoint the cause of th writer's problem.
If the industry as a whole is undergoing poor revenues, then perhaps the issue is not Super Screen's advertising company but rather the country's economy.
In times of economic \emph{strife}, it is only natural that people would be less willihng to spend money on luxuries such as movie tickets.
If this is the case, it might be better suit the production company to cut costs rather than refunneling them to a different part of the company.

Second, the advertising director should ask himself this: what medium do the majority of his most generous movie reviewers utilize?
The writer states that movie reviews were generally positive, but where were these reviews located?
On television, newspapers, or the Internet?
It is possible that the medium used by the most movies reviewers of Super Screen's movies is one that is not utilized by most of the company's target audiences.
If Super Screen produce many family films, but most of the good reviews are found in late night television shows, then there is a good chance that the reviews are not going to be seen by the target audience.
If this is the case, then the company would be better off conducting research as to what medium is most likely to reach their audiences.

One last question would be this: what advertising is currently being used by the Super Screen company?
If the company advertises using only one medium, such as in newspapers, perhaps the solution is not to double the amount of newspaper space but to branch out and try other forms of advertising.
The writer fails to mention exactly how the company currently advertises their movies, and this absence detracts from his argument.

In conclusion, the advertising director would be better served by first answering thse questions and evaluating the resulting answers before pouring millions of dollars into his solution.
It is possible that an alternative solution exists, perhaps one that will not be as expensive nor as risky.