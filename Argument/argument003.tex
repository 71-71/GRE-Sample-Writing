\section{Argument 3}

\paragraph{The following appeared in a memorandum from the owner of Movies Galore, a chain of video rental stores.}
\subparagraph{In order to reverse the recent decline in our profits, we must reduce operating expenses at Movies Galore's ten video rental stores.
Since we are famous for our special bargains, raising our rental prices is not a viable way to improve profits.
Last month our store in downtown Marston significantly decreased its operating expenses by closing at 6:00 P.M. rather than 9:00 P.M. and by reducing its stock by eliminating all movies released more than five years ago.
Therefore, in order to increase profits without jeopardizing our reputation for offering great movies at low prices, we recommend implementing similar changes in our other nine Movies Galore stores.}

\subsection{Score 6}
One question which needs to be addressed before implementing the recommendation is whether there are not other ways to improve profits besides cutting operating expenses.
Without proof, the author decides, first, that there are only two viable options for increasing the profits of Movie Calore: raising rental prices, and cutting costs.
He rules out the first course, and hence claims the second option must be chosen.
But it seems there may be alternative methods of increasing profits, such as initiating advertising campaigns or closing unprofitable Movie Galore locations.

Even if it is granted that there are only two options for increasing profitability--cutting costs, and raising rental prices--one might wonder why raising rental prices is so unthinkable.
The author implies that because Movie Galore is famous for special bargains, raising the rental prices would eliminate this competitive advantage and decrease profitability.
However, in making this conclusion, he makes several assumptions without considering questions that need to be addressed.
First, he assumes that there is no room to raise current prices and yet maintain lower prices than competitors.
One would need to ask if prices could be increased slightly, which keeping them cheap.
Even if there is no room for such strategy, the author assumes that Movies Galore's reputation for bargain pricing would evaporate if they increased their prices slightly.
Perhaps such a reputation would be widespread enough to persist despite a slight increase in prices.
And thirdly, even if the reputation for bargains would be eliminated by an increase in prices, the author assumes that Movies Galore cannot change course and be successful in some other way.
Perhaps it could instead become known as the store with the friendliest employees.
Perhaps it already it, and the author is wrong to believe that a causal relationship between bargain prices and success exists, when the real cause of Movies galore's good reputation is entirely independent of its prices.
The author needs to answer these questions to convince us that profits are caused by bargains, and not by other factors that may be involved.

Another question that needs to be raised is whether or not the downtown Marston store is truly analogous to the other nine Movies Galore stores.
The author seems to assume that because the cost-cutting measures worked at the Marston location, it will work at the others, but this is far from clear.
Perhaps the patrons of the other Movie Galore locations would resent such changes in the hours and stock of their local stores.
Perhaps the most important question that needs to be asked is whether the Marston location's changes truly increased profitability.
The author writes that the Marston store decreased operating expenses by closing earlier and cutting its stock, but he makes no mention of increased profitability.
It is quite possible that the Marston location's profits decreased as a result of their cost cutting, and this is a question that needs to be addressed.
The author then jumps to the conclusion that taking similar measures would increase profitability at other locations, though such a connection has not even been established at the Marston store.

Even if the cost-cutting measures increased profitability at the Marston store last month (and a causal relationship, though presumably assumed, is still far from evident), there is no guarantee that such measures would continue to increase profitability over time.
One would need to ask: why not observe how the Marston location's action affect profitability over several months, before implementing such sweeping changes at every store?
A single month is a very short time span, and the habits of customers may change slowly.
As word gets around that the Marston store has cut their hours and their selection, they may in fact jeopardize their reputation for offering ``great movies at low prices''.
After all, the name of the franchise is Movies Galore, and by drastically reducing the available selection, they may alienate their customer base.
If, as mentioned above, Movies Galore is famous for more than its great bargains--if customers prefer Movies Galore because of its selections, as well--then such a move may drastically reduce profits over time.
It seems extremely rash to implement such a new and relatively untried strategy at every Movies Galore location, before the effect can be observed and interpreted.

\subsection{Score 5}
Management's prediction that declining profits could be reversed by reducing operating hours and reducing stocks seems to be rash since there is little evidence that proper research has been conducted.
It may be true that profits could be restored by cutting operating costs, but management needs to ask whether making these changes would have a negative impact on its best source of revenue.

The management states that downtown Marston store ``significantly decreased its operating expenses by closing at 6:00 p.m. rather that 9:00 p.m.''
It is reasonable to think that closing at 6:00 p.m. rather than 9:00 p.m. would decrease operating expenses, but the business is concerned with renting movies and these may be the busiest and most profitable hours of operation.
Could it be that most people renting movies have normal working hours and have leisure time at night and to fill that time they turn to renting movies?
If management researches its daily rental history, it may discover that its peak rental hours are between 6:00 p.m. and 9:00 p.m.
It this is the case, the store could lose significant cincome or even go out of business altogether.
If management wants to reverse a decline in profits by cutting hours of operation and thereby reducing expenses, it would be adventageous to determine through research which block of time during the day is the least profitable and then cut those hours of operation.
For instance, if it is found that profits are lowest during the morning hours and around noon, it would be better to close the store during those hours rather than during the hours that bring in the greatest profits.

The management then states that operating expenses will also be cut ``by eliminating all movies released more than five years ago''.
Again, more research is needed in order to determine if this would indeed help reverse the decline in profits that Movies Galore is experiencing.
Is it possible that the success of a movie rental business is based on its ability to provide customers with a wide array of movie selections, both new and old?
It could be dangerous for this business to eliminate its stock of older movies without first determining the percentages of income that come from each product.
Management should research its history of movie rentals in order to determine if a significant percentage of its profits come from the rental of older movies.
Even if little profit does come from older movies, it may still be unwise to eliminate the stock of old movies.
If Movies Galore maintains a variety of movies, a person searching for a current movie may decide to rent an older movie as well.
This may be especially relevent in the case of a new movie that is sequal to an older movie or part of a trilogy.
Reducing movie variety may also damage the reputation of the stores.
The management states that Movie Galore already has a ``reputation for offering great movies''.
If movie variety is suddenly reduced, the stores may gain a negative reputation.

Overall, the management makes a prediction that is untrustworthy and potentially damaging.
More research should be conducted to see if indeed such changes to cut operation costs will be effective, and if not, what should be done instead.
If the proper investigation is implemented by the management, Movie Galore store may reverse the recent decline of profits.